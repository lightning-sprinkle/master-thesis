\chapter*{Abstract}
\label{cha:abstract}
The business of online advertising has evolved in a way that is not transparent anymore. A handful of large advertisement firms are controlling practically every online ad you see and collecting data about your browsing behavior. In the meantime, several other revenue models for online content are implemented, such as paywalls and asking for donations. However, these structures are affecting the browsing experience and are way more expensive than the revenue that comes from online advertising. 

In this thesis, a concept that features an automatic universal in-browser payments system is presented. This system sends small payments to the publishers of the websites that are visited in the browser. The value of the payments is comparable to the revenue that online advertising would generate.

In order to proof this concept, this thesis also features an implementation of a working prototype. The prototype makes use of the Lightning Network, which is an extra layer on the Bitcoin blockchain so that micropayments can be facilitated.

Lastly, this thesis discusses the societal impact of such a system and reviews what a web browsing experience without advertising could bring. 

\chapter*{Zusammenfassung}
\label{cha:abstract}
Online-Werbung hat sich in einer nicht-transparenten Weise entwickelt. Eine Handvoll großer Werbeunternehmen kontrolliert praktisch jede Online-Werbung, die man sieht, und sammelt Daten über das Surfverhalten. In der Zwischenzeit werden verschiedene andere Business Models für Online-Content eingeführt, wie Paywalls und Spendenaufrufe. Diese Strukturen beeinträchtigen jedoch das Surferlebnis und sind weitaus teurer als die Einnahmen aus der Online-Werbung.

In dieser Arbeit wird ein Konzept vorgestellt, das ein automatisches universelles In-Browser-Zahlungssystem vorsieht. Dieses System sendet kleine Zahlungen an die Herausgeber der Websites, die im Browser besucht werden. Der Wert der Zahlungen ist vergleichbar mit den Einnahmen, die durch Online-Werbung erzielt würden.

Um dieses Konzept zu überprüfen, wird in dieser Arbeit auch die Implementierung eines funktionierenden Prototyps vorgestellt. Der Prototyp nutzt das Lightning Network, das eine zusätzliche Schicht auf der Bitcoin-Blockchain darstellt, sodass Mikrozahlungen erleichtert werden können.

Schließlich diskutiert diese Arbeit die gesellschaftlichen Auswirkungen eines solchen Systems und untersucht, was ein Surfen im Internet ohne Werbung bringen könnte.